\documentclass[]{article}
\usepackage[]{graphicx}
\usepackage[]{xcolor}
% maxwidth is the original width if it is less than linewidth
% otherwise use linewidth (to make sure the graphics do not exceed the margin)
\makeatletter
\def\maxwidth{ %
  \ifdim\Gin@nat@width>\linewidth
    \linewidth
  \else
    \Gin@nat@width
  \fi
}
\makeatother

\usepackage{amsmath}

\newcommand{\logictable}[2]{
  \begin{tabular}{|p{4cm}|p{10cm}|}
    \hline
    \textbf{Triggered by} & #1\\ \hline
    \multicolumn{2}{|c|}{\parbox{14cm}{#2}} \\ \hline
  \end{tabular}
}
\newcommand{\simTime}{\textsc{time}}

\usepackage{algpseudocode}
\usepackage{array}
\usepackage{booktabs}
\usepackage{arydshln}
\usepackage{multirow}

\usepackage{bold-extra}
\usepackage{hyperref}
\usepackage{tikz}
%\usepackage{amsmath, amssymb,bm,color}
\usetikzlibrary{shapes,arrows}
\usetikzlibrary{calc}
\usetikzlibrary{positioning}

\usepackage[normalem]{ulem}

\usepackage{geometry}
\geometry{margin=1 in}
\setlength\parindent{0pt}

\usepackage{algpseudocode}
\newlength{\firstcol}
\newlength{\secondcol}
\newlength{\thirdcol}
\newlength{\fourthcol}
\newlength{\fifthcol}
\newlength{\lastcol}
\newlength{\agap}
\setlength{\agap}{3pt}
\newcommand{\gapit}[1]{\vspace*{\agap}#1\vspace*{\agap}}
\algrenewcommand\algorithmicindent{1cm}
\newlength{\algoff}
\setlength{\algoff}{-0.18cm}
\algrenewcommand{\algorithmiccomment}[1]{%
\textit{$//$ #1}%
%\textit{$\#$ #1}
}

\usepackage{longtable}

\makeatletter
\newcommand{\StatexIndent}[1][3]{%
  \setlength\@tempdima{\algorithmicindent}%
  \Statex\hskip\dimexpr#1\@tempdima\relax}
\algdef{S}[IF]{IfNoThen}[1]{\algorithmicif\ #1}%
\makeatother


\input{ActivityDiagrams/activity_diagram_styles.tex}

\title{Conceptual Modelling Lab 2}
\IfFileExists{upquote.sty}{\usepackage{upquote}}{}
\begin{document}

\vspace*{-0.08\textheight}

\section*{Conceptual Modelling Assignment -- $<$Name, UPI$>$}

\subsection*{Problem Understanding}

\vspace*{300pt}

\subsection*{Identification of Modelling and General Objectives}

\subsubsection*{Modelling Objectives}

\vspace*{64pt}

\subsubsection*{General Objectives}

\vspace*{32pt}

\subsection*{Defining Input Factors}

\vspace*{32pt}

\subsection*{Defining Output Responses}

\vspace*{32pt}

\subsection*{Model Content}

\subsection*{Identifying Entities}

\vspace*{16pt}

\newpage

\subsection*{Drawing Behavioural Paths}

\begin{figure}[htp]
\caption{Person Behavioural Path}
\begin{tikzpicture}
\node [inout] (v1) at (0,0) {Person Arrives};

\draw [_sched] (v1) to node[above] {P.T1} (5, 0);
\end{tikzpicture}
\vspace*{160pt}
\end{figure}

\begin{figure}[htp]
\caption{Turnstile Behavioural Path}
\begin{tikzpicture}
\node [inout] (v1) at (0,0) {Turnstile Created};

\draw [_sched] (v1) to node[above] {T.T1} (5,0);
\end{tikzpicture}
\end{figure}

\subsection*{Model Control -- Defining Logic}

\textbf{Logic} \\[6pt]
\logictable{Entity}{
\begin{algorithmic}[1]
\State
\State
\State
\State
\State
\State
\State
\State
\end{algorithmic}
} \\[6pt]
\textbf{Logic} \\[6pt]
\logictable{Entity}{
\begin{algorithmic}[1]
\State
\State
\State
\State
\State
\State
\State
\State
\end{algorithmic}
} \newpage
\textbf{On Start Choose Direction} \\[6pt]
\logictable{Person P}{
\begin{algorithmic}[1]
\If{P.CurrentSection = P.Section}
  \State Choose Direction.End with P
  \State TRANSITION P.T6 Choose Direction.End to Person Seated with P
\Else
\If{P.FirstChoice}
  \State 
\ElsIf{(P.CurrentSection = ``A'') AND (P.Direction = ``down'')}
  \StatexIndent[2] \Comment{Crossing ``A'' to ``X'', find best direction from distance}
  \If{(P.Section - ``A'') $<$ (``X'' - P.Section)}
    \State P.Direction = ``up''
  \Else
    \State P.Direction = ``down''
  \EndIf
\ElsIf{(P.CurrentSection = ``X'') AND (P.Direction = ``up'')}
  \StatexIndent[2] \Comment{Crossing ``X'' to ``A'', find best direction from distance}
  \If{(P.Section - ``A'') $>$ (``X'' - P.Section)}
    \State P.Direction = ``down''
  \Else
    \State P.Direction = ``up''
  \EndIf
\Else
  \If{}
    \State 
  \Else \Comment{}
    \State 
  \EndIf
\EndIf
\If{}
\State
\Else
\State
\EndIf
\EndIf
\end{algorithmic}
}

\subsection*{Model Data}

\begin{table}[htp]
\centering
\begin{tabular}{
|
>{\centering\arraybackslash}p{2.5cm}|
>{\centering\arraybackslash}p{2.25cm}|
>{\centering\arraybackslash}p{2.5cm}|
>{\centering\arraybackslash}p{2.25cm}|
>{\centering\arraybackslash}p{2.25cm}|}
\hline

\textbf{Data}
& 
\textbf{Source}
&
\textbf{Identification}
&
\textbf{Input}
& 
\textbf{Output} \\ \hline
 &  &  &  & 
\\ \hline
 &  &  &  & 
\\ \hline
 &  & & & Function of $N$ (defined below)
\\ \hline
\end{tabular}

\[
T=\begin{cases} 60 & N < 100 \\
59 + 1000^{675\times10^{-6} (N-100)} & N \geq 100
\end{cases}
\]

\end{table}

\subsection*{Model Entities}

Note that default attributes CurrentStart and CurrentActivity are omitted for brevity.

\setlength{\firstcol}{1.8cm}
\setlength{\secondcol}{1.4cm}
\setlength{\thirdcol}{1.4cm}
\setlength{\fourthcol}{2cm}

\begin{table}[htp]
  \centering

\begin{tabular}{|p{\thirdcol}|p{\fourthcol}|p{5cm}|}
\hline
  \multirow{6}{*}{} &
  \parbox{\fourthcol}{\vspace*{3pt}\centering\textbf{Type}} &
  \\ \cline{2-3} 
                            & \multirow{5}{*}{\parbox{\fourthcol}{\centering\textbf{Attributes  -- default value or range in []}}} &  []                                   \\ \cline{3-3} 
                            &                                                                                      &  []                                         \\ \cline{3-3} 
                            &                                                                                      &  []                                  \\ \cline{3-3} 
                            &                                                                                      &  []                          \\ \cline{3-3} 
                            &                                                                                      &  []                            \\ \cline{1-3} 
  \multirow{3}{*}{} &
  \parbox{\fourthcol}{\vspace*{3pt}\centering\textbf{Type}} &
   \\ \cline{2-3} 
                            & \multirow{2}{*}{\parbox{\fourthcol}{\centering\textbf{Attributes}}} & [] \\ \cline{3-3} 
                            &                                                                                      &  []                                  \\ \cline{1-3} 
  \multirow{3}{*}{Section} &
  \parbox{\fourthcol}{\vspace*{3pt}\centering\textbf{Type}} &
   Passive\\ \cline{2-3} 
                            & \multirow{2}{*}{\parbox{\fourthcol}{\centering\textbf{Attributes}}} & ID [A-X] \\ \cline{3-3} 
                            &                                                                                      &  []                                  \\ \cline{1-3} 
  \multirow{2}{*}{} &
  \parbox{\fourthcol}{\vspace*{3pt}\centering\textbf{Type}} &
   \\ \cline{2-3} 
                            & \multirow{1}{*}{\parbox{\fourthcol}{\centering\textbf{Attributes}}} & [] \\ \hline
\end{tabular}%

\end{table}



\vspace*{-16pt}

\subsection*{Model Transitions}

\setlength{\firstcol}{2cm}
\setlength{\secondcol}{0.75cm}
\setlength{\thirdcol}{3.7cm}
\setlength{\fourthcol}{6cm}
\setlength{\fifthcol}{6cm}

\begin{table}[htp]
\centering

\begin{tabular}{|c|c|c|}
\hline
\textbf{Transitions} & \textbf{From Event} & 
\textbf{To Event} \\
\hline
   &  & \\ \cline{1-3}
   &  & \\ \cline{1-3}
   &  & \\ \cline{1-3}
   &  & \\ \cline{1-3}
   &  & \\ \cline{1-3}
   &  & \\ \cline{1-3}
   &  & \\ \cline{1-3}
   &  & \\ \cline{1-3}
   &  & \\ \hline 
 \end{tabular}%
\end{table}


\subsection*{Model Activities}

\setlength{\firstcol}{0.75cm}
\setlength{\secondcol}{0.75cm}
\setlength{\thirdcol}{1.6cm}
\setlength{\fourthcol}{1.5cm}
\setlength{\fifthcol}{1.7cm}
\setlength{\lastcol}{9cm}

\makeatletter
\def\@cline#1-#2\@nil{%
  \omit
  \@multicnt#1%
  \advance\@multispan\m@ne
  \ifnum\@multicnt=\@ne\@firstofone{&\omit}\fi
  \@multicnt#2%
  \advance\@multicnt-#1%
  \advance\@multispan\@ne
  \leaders\hrule\@height\arrayrulewidth\hfill
  \cr
  \noalign{\nobreak\vskip-\arrayrulewidth}}
\makeatother

\begin{tabular}{|p{\thirdcol}|p{\fourthcol}p{\fifthcol}|p{\lastcol}|}
\hline
  \multirow{8}{*}{\parbox{\thirdcol}{\centering }} &
  \multicolumn{2}{c|}{\parbox{\fourthcol}{\gapit{\centering\textbf{Participants}}}} &
  \parbox{\lastcol}{
 }
 \\* \cline{2-4} 
   &
  \multicolumn{1}{c|}{\multirow{3}{*}{\parbox{\fourthcol}{\centering\textbf{Start Event}}}} &
  \parbox{\fifthcol}{\gapit{\centering\textbf{Type}}} &
   \\* \cline{3-4} 
   &
  \multicolumn{1}{c|}{} &
 \parbox{\fifthcol}{\gapit{\centering\textbf{State Change}}} &
  \hspace*{\algoff}\parbox{\lastcol}{
  \begin{algorithmic}[1]
    \State 
  \end{algorithmic}
  } \\* \cline{2-4} 
   &
  \multicolumn{1}{c|}{\multirow{3}{*}{\parbox{\fourthcol}{\centering\textbf{End Event }}}} &
  \parbox{\fifthcol}{\gapit{\centering\textbf{Type}}} &
   \\* \cline{3-4} 
   &
  \multicolumn{1}{c|}{} &
  \parbox{\fifthcol}{\gapit{\centering\textbf{State Changes}}} &
  \hspace*{\algoff}\parbox{\lastcol}{
  \begin{algorithmic}[1]
      \State 
      \State \Comment{TRANSITION ??? is determined by logic}
  \end{algorithmic}
    }\\ \cline{1-4} 
   \multirow{10}{*}{\parbox{\thirdcol}{\centering }} &
  \multicolumn{2}{c|}{\parbox{\fourthcol}{\gapit{\centering\textbf{Participants}}}} &
  \parbox{\lastcol}{
  
  } \\* \cline{2-4} 
   &
  \multicolumn{1}{c|}{\multirow{3}{*}{\parbox{\fourthcol}{\centering\textbf{Start Event}}}} &
  \parbox{\fifthcol}{\gapit{\centering\textbf{Type }}} &
   \\* \cline{3-4} 
   &
  \multicolumn{1}{c|}{} &
  \parbox{\fifthcol}{\gapit{\centering\textbf{State Change}}} &
  \hspace*{\algoff}\parbox{\lastcol}{
  \begin{algorithmic}[1]
  \State 
  \end{algorithmic}} \\* \cline{2-4} 
   &
  \multicolumn{1}{c|}{\multirow{4}{*}{\parbox{\fourthcol}{\centering\textbf{End Event}}}} &
  \parbox{\fifthcol}{\gapit{\centering\textbf{Type}}} &
   \\* \cline{3-4} 
   &
  \multicolumn{1}{c|}{} &
  \parbox{\fifthcol}{\gapit{\centering\textbf{State Changes}}} &
  \hspace*{\algoff}\parbox{\lastcol}{
  \begin{algorithmic}[1]
      \State 
      \State 
      \State 
  \end{algorithmic}
  } \\ \hline 
\end{tabular}

\begin{tabular}{|p{\thirdcol}|p{\fourthcol}p{\fifthcol}|p{\lastcol}|}
  \hline
  \multirow{7}{*}{\parbox{\thirdcol}{\centering }} &
  \multicolumn{2}{c|}{\parbox{\fourthcol}{\gapit{\centering\textbf{Participants}}}} &
  \parbox{\lastcol}{
  } \\* \cline{2-4} 
   &
  \multicolumn{1}{c|}{\multirow{2}{*}{\parbox{\fourthcol}{\centering\textbf{Start Event}}}} &
  \parbox{\fifthcol}{\gapit{\centering\textbf{Type}}} &
  \\* \cline{3-4} 
   &
  \multicolumn{1}{c|}{} &
  \parbox{\fifthcol}{\gapit{\centering\textbf{State Changes}}} &
  \hspace*{\algoff}\parbox{\lastcol}{
  \begin{algorithmic}[1]
  \State 
  \end{algorithmic}} \\* \cline{2-4} 
   &
  \multicolumn{1}{c|}{\multirow{2}{*}{\parbox{\fourthcol}{\centering\textbf{End Event}}}} &
  \parbox{\fifthcol}{\gapit{\centering\textbf{Type}}} &
   \\* \cline{3-4} 
   &
  \multicolumn{1}{c|}{} &
  \parbox{\fifthcol}{\gapit{\centering\textbf{State Changes}}} &
  \hspace*{\algoff}\parbox{\lastcol}{
  \begin{algorithmic}[1]
      \State \Comment{TRANSITION ??? or ??? determined by logic}
  \end{algorithmic}
  } \\ \cline{1-4} 
  \multirow{10}{*}{\parbox{\thirdcol}{\centering }} &
  \multicolumn{2}{c|}{\parbox{\fourthcol}{\gapit{\centering\textbf{Participants}}}} &
  \parbox{\lastcol}{
  }
\\* \cline{2-4} 
   &
  \multicolumn{1}{c|}{\multirow{3}{*}{\parbox{\fourthcol}{\centering\textbf{Start Event}}}} &
  \parbox{\fifthcol}{\gapit{\centering\textbf{Type}}} &
  Controlled \\ \cline{3-4} 
   &
  \multicolumn{1}{c|}{} &
  \parbox{\fifthcol}{\gapit{\centering\textbf{State Changes}}} &
  \hspace*{\algoff}\parbox{\lastcol}{
  \begin{algorithmic}[1]
  \State 
  \State 
  \end{algorithmic}} \\* \cline{2-4} 
   &
  \multicolumn{1}{c|}{\multirow{4}{*}{\parbox{\fourthcol}{\centering\textbf{End Event}}}} &
  \parbox{\fifthcol}{\gapit{\centering\textbf{Type}}} &
   \\* \cline{3-4} 
   &
  \multicolumn{1}{c|}{} &
  \parbox{\fifthcol}{\gapit{\centering\textbf{State Changes}}} &
  \hspace*{\algoff}\parbox{\lastcol}{
  \begin{algorithmic}[1]
  \State 
  \State
  \State 
  \end{algorithmic}
  } \\ \cline{1-4} 
  \multirow{7}{*}{\parbox{\thirdcol}{\centering }} &
  \multicolumn{2}{c|}{\parbox{\fourthcol}{\gapit{\centering\textbf{Participants}}}} &
  \parbox{\lastcol}{
  
 }
\\* \cline{2-4} 
   &
  \multicolumn{1}{c|}{\multirow{3}{*}{\parbox{\fourthcol}{\centering\textbf{Start Event}}}} &
  \parbox{\fifthcol}{\gapit{\centering\textbf{Type}}} &
   \\* \cline{3-4} 
   &
  \multicolumn{1}{c|}{} &
 \parbox{\fifthcol}{\gapit{\centering\textbf{State Change}}} &
  \hspace*{\algoff}\parbox{\lastcol}{
  \begin{algorithmic}[1]
    \State 
  \end{algorithmic}
  } \\* \cline{2-4} 
   &
  \multicolumn{1}{c|}{\multirow{3}{*}{\parbox{\fourthcol}{\centering\textbf{End Event }}}} &
  \parbox{\fifthcol}{\gapit{\centering\textbf{Type}}} &
  Controlled \\* \cline{3-4} 
   &
  \multicolumn{1}{c|}{} &
  \parbox{\fifthcol}{\gapit{\centering\textbf{State Changes}}} &
  \hspace*{\algoff}\parbox{\lastcol}{
  \begin{algorithmic}[1]
      \State
      \State \Comment{}
  \end{algorithmic}
  }  \\ \hline
\end{tabular}


\subsection*{Model Events}

\setlength{\firstcol}{0.75cm}
\setlength{\secondcol}{0.75cm}
\setlength{\thirdcol}{1.6cm}
\setlength{\fourthcol}{2.5cm}
\setlength{\fifthcol}{9cm}

\begin{tabular}{|p{\thirdcol}|p{\fourthcol}|p{\fifthcol}|}
\hline
  \multirow{9}{*}{\parbox{\thirdcol}{\centering Simulation Start}} &
  \parbox{\fourthcol}{\vspace*{3pt}\centering\textbf{Participant}} &
  None \\ \cline{2-3} 
   &
  \parbox{\fourthcol}{\vspace*{3pt}\centering\textbf{Type}} &
  \\ \cline{2-3} 
   &
  \parbox{\fourthcol}{\vspace*{3pt}\centering\textbf{State Changes}} &
  \hspace*{\algoff}\parbox{\fifthcol}{
  \begin{algorithmic}[1]
      \State
      \State
      \State
      \State
      \State
  \end{algorithmic}
  }  
  \\ \hline
\multirow{8}{*}{\parbox{\thirdcol}{\centering Person Arrives}} &
  \parbox{\fourthcol}{\vspace*{3pt}\centering\textbf{Participant }} &
  Person (P), Turnstile (T) \\ \cline{2-3} 
   &
  \parbox{\fourthcol}{\vspace*{3pt}\centering\textbf{Type }} &
  \\ \cline{2-3} 
   &
  \parbox{\fourthcol}{\vspace*{3pt}\centering\textbf{State Changes }} &
\hspace*{\algoff}\parbox{\fifthcol}{  
  \begin{algorithmic}[1]
    \State 
    \State 
    \State
    \State
  \end{algorithmic}
} \\ \hline
\end{tabular}

\begin{tabular}{|p{\thirdcol}|p{\fourthcol}|p{\fifthcol}|}
\hline
\multirow{3}{*}{\parbox{\thirdcol}{\centering }} &
  \parbox{\fourthcol}{\vspace*{3pt}\centering\textbf{Participant }} &
  \\ \cline{2-3} 
   &
  \parbox{\fourthcol}{\vspace*{3pt}\centering\textbf{Type }} &
  \\ \cline{2-3} 
   &
  \parbox{\fourthcol}{\vspace*{3pt}\centering\textbf{State Changes}} &
\hspace*{\algoff}\parbox{\fifthcol}{  
  \begin{algorithmic}[1]
    \State 
  \end{algorithmic}
}    
  \\ \hline
\multirow{7}{*}{\parbox{\thirdcol}{\centering }} &
  \parbox{\fourthcol}{\vspace*{3pt}\centering\textbf{Participant }} &
   \\ \cline{2-3} 
   &
  \parbox{\fourthcol}{\vspace*{3pt}\centering\textbf{Type }} &
  \\ \cline{2-3} 
   &
  \parbox{\fourthcol}{\vspace*{3pt}\centering\textbf{State Changes }} &
\hspace*{\algoff}\parbox{\fifthcol}{  
  \begin{algorithmic}[1]
    \State T.ID = max(U.ID for U in Turnstiles) + 1
    \Statex \Comment{Get next ID}
    \State 
    \State 
  \end{algorithmic}
}
  \\ \hline
\multirow{5}{*}{\parbox{\thirdcol}{
\centering 
Simulation Finish}} &
  \parbox{\fourthcol}{\vspace*{3pt}\centering\textbf{Participant }} &
  None \\ \cline{2-3} 
   &
  \parbox{\fourthcol}{\vspace*{3pt}\centering\textbf{Type }} &
  Scheduled \\ \cline{2-3} 
   &
  \parbox{\fourthcol}{\vspace*{10pt}\centering\textbf{State Changes}} &  
  \vspace*{-16pt}
  \begin{algorithmic}[1]
      \For{T $\in$ Turnstiles}
      \State Calculate statistics for T
      \EndFor
  \end{algorithmic}
  \vspace*{3pt}
  \\ \hline 
\end{tabular}

 
 
 

\end{document}

